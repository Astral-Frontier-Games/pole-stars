% !TEX engine = xelatex
\documentclass[10pt,oneside,letterpaper,landscape]{memoir}

% set your margins here; left margin, right margin, bottom margin, top margin, binding offset, etc
\usepackage[lmargin=13mm,rmargin=13mm,bmargin=11mm,tmargin=13mm,bindingoffset=0mm,heightrounded]{geometry}

% these are used for placing text outside the natural flow of text, in blocks and boxes (and even rotated)
\usepackage[absolute]{textpos}
\setlength{\TPHorizModule}{10mm}
\setlength{\TPVertModule}{10mm}
\usepackage{rotating}

% render lists in more compact form
\usepackage{enumitem}
\setlist[enumerate]{noitemsep, topsep=-2pt}
\setlist[itemize]{noitemsep, topsep=-2pt}

% this allows you to use images
\usepackage{graphicx}
\graphicspath{ {images/} }

% font stuff
\usepackage{fontspec,xltxtra,xunicode}
\defaultfontfeatures{Mapping=tex-text}

% main font, used for the full document; use the family name here. If scale = 1 it is 10pt; you may want to go to .9 or even .8
\setmainfont[Scale=1,Path=./fonts/,BoldFont={AlegreyaSans-Bold},ItalicFont={AlegreyaSans-Italic}]{AlegreyaSans-Regular}

% each of these is a font we'll use later; you can add more and scale them using the same method
% you may want more or less of these; at minimum, you want a title font and a section header font
% you'll need to mess with scale on these to get them looking 'nice' for your fonts
\newfontfamily\titlefont[Scale=2.5,Path=./fonts/,BoldFont={Montserrat-Bold},ItalicFont={Montserrat-Italic}]{Montserrat-Regular}
\newfontfamily\sectionfont[Scale=1,Path=./fonts/,BoldFont={Montserrat-Bold},ItalicFont={Montserrat-Italic}]{Montserrat-Regular}

% this is for the purposes of Lettrine
\newfontfamily\numeralfont[Scale=1,Path=./fonts/,BoldFont={Montserrat-Bold},ItalicFont={Montserrat-Italic}]{Montserrat-Regular}

% new commands; these allow us to set up references to our fonts and format them in advance
\newcommand*\toptitle[1]{{\color{titlecolor}\titlefont{#1}}}
\newcommand*\tagline[1]{{\color{taglinecolor}\taglinefont{#1}}}

% this is a seperator, like a dagger or bullet, for inline lists
\newcommand*\sep[0]{\* }

% this lets us use multiple columns
\usepackage{multicol}
% this sets how much space between any two columns
\setlength{\columnsep}{2em}

% this is for tables
\usepackage{tabularx}

% and here's where we define colors, giving each a name so we can refer to it
\usepackage{xcolor}

% first basic stuff 
\definecolor{offwhite}{HTML}{ffffff}

% now single use case items like 'text I put in the margins'
\definecolor{margintext}{HTML}{363b4d}

% ... and box corners/backgrounds
\definecolor{plainboxbg}{HTML}{F3F3F3}
\definecolor{plainboxtxt}{HTML}{000000}
\definecolor{plainboxtabs}{HTML}{363b4d}

% ... and the title color
\definecolor{titlecolor}{HTML}{363b4d}
\definecolor{taglinecolor}{HTML}{363b4d}

% ... and section header color
\definecolor{shadecolor}{HTML}{909090}

% this is just a convenience function to put a box around text
\newcommand{\shadebox}[1]{\par\noindent\colorbox{shadecolor}
{\parbox{\dimexpr\textwidth-2\fboxsep\relax}{#1}}\smallskip}

% this puts a repeating symbol over and over to the left of the text; an alternate header
\newcommand\rep{\leavevmode\xleaders\hbox{/ }\hfill\kern0pt}

% this creates fancier boxes
\usepackage[raster, skins, hooks, many]{tcolorbox}

% ... but we're just creating a simple one to start
\newtcolorbox{plainbox}[1][]{fonttitle=\color{white}\headfont, colframe=shadecolor, colback=white, after skip=2mm, #1}

% drop caps, very useful for mechanic results
\usepackage{lettrine}
\renewcommand{\LettrineFontHook}{\numeralfont}
\setlength{\DefaultNindent}{2pt}
\setlength{\DefaultFindent}{2pt}

% uncomment this to change the entire page's background
%\pagecolor{\color{50!red}}

% hide links so they're not marked out
\usepackage[hidelinks]{hyperref}

% and now some general formatting; you can adjust other parts by changing 'sec' to 'subsec' or 'subsubsec'

% remove numbering from section titles, as we'll use our sections as dividers between rules chunks
\setsecnumformat{}

% now reduce spacing around section tags
\setaftersecskip{1mm}
\setbeforesecskip{3mm}

% and redefine what font we use for section heads
\setsecheadstyle{\sectionfont\color{blue}}% Set \section style

% save the current indent so you can use \indent when you want to indent
\newlength\tindent
\setlength{\tindent}{\parindent}
\renewcommand{\indent}{\hspace*{\tindent}}

% and here we set the default indent to nothing and a line between paragraphs instead
\setlength{\parindent}{0pt}
\setlength{\parskip}{.5\baselineskip}

% make it so text doesn't squash down to the bottom
\raggedbottom

% now remove page numbering since this is just one sheet
\pagestyle{empty}

% this redefines minipage (basically an invisible box) to use all the same settings as above
\newlength{\currentparskip}
    \newenvironment{mpage}[1]{%
        \setlength{\currentparskip}{\parskip}% save the value
             \begin{minipage}{#1}% open the minipage
         \setlength{\parskip}{\currentparskip}% restore the value
    }
{\end{minipage}}

% and this sets up a double rule you can use as a divider
\newcommand{\doublerule}[1][.4pt]{%
  \noindent\color{black!75}
  \makebox[0pt][l]{\rule[.7ex]{\linewidth}{#1}}%
  \rule[.3ex]{\linewidth}{#1}\color{black}\vspace{-1mm}}

%%% BEGIN DOCUMENT; this is required
\begin{document}

% set up 3 columns; this will be matched at the very end with \end{multicols}
\begin{multicols}{3}

%  I like to set up any textblocks (from textpos package) up front. The 13 is how wide the textblock is, and the 13.5 and 21 tell you how high on the page and how far over it is. Try setting it to 0,0 to see the difference.
  \begin{textblock}{15}(13.5,21)
      \footnotesize
        \textit{\href{https://github.com/Astral-Frontier-Games/pole-stars}{Pole Stars} is licensed under a \href{https://creativecommons.org/licenses/by/4.0/}{Creative Commons Attribution 4.0 International License}.}
  \end{textblock}

% this is how you add another textblock, and how you include a graphic
\begin{textblock}{18}(15.25,0.5)
  \includegraphics[width=.15\textwidth]{images/StarSilhouette.png}
\end{textblock}

% now we start laying it out

%vspace and hspace adjust the vertical space above their line and the horizontal space before it, respectively
% the asterisk says 'no really, do this' if latex for some reason won't

% first, open a mpage, our custom minipage

% columnwidth is the width of one column, in this case, a third of the page. By putting a number in front of it, you can multiply it by that number 
% two columns is roughly 2.13. This stretches the invisible block across two columns.
\begin{mpage}{2.13\columnwidth}

 \hspace*{0mm}{\toptitle{Pole Stars}}

% a little extra space over what'd normally be provided
\vspace{3mm}

\textit{A system-agnostic supplement for creating character relationships and campaign issues.}

\doublerule

\end{mpage}

% a little extra space over what'd normally be provided
\vspace{2mm}

\section{What is a Pole Star?}

A \textit{pole star} is a fixed point around which character drama can revolve. Pole stars are things a character wants to do something about, or with. They encourage characters to have a reason to be together and interact, but they may also introduce tension or drama. Pole stars aren't what your game is "about", but they can be if you want.

A pole star is an element of the fictional game world, like "rune magic" or "dragons" or "the Northern invasion". If you suggest a pole star, you are asking to add it to the world. The group must decide whether a given pole star is appropriate or not, using safety tools and personal preferences.

Characters will have a \textit{position} toward a pole star. If you can say "My character wants to (something) this pole star", or "My character is (something) about/toward this pole star", that something is your character's position.

\section{Using Pole Stars}

The group should decide on one or two pole stars at the start of a game, during character creation. You don't have to flesh out every detail of a pole star, but everyone should understand and agree what is meant by a pole star. For example, you might pick "rune magic". You don't have to know all the rules of how rune magic works, only that it exists.

Players should come up with positions for their characters, then share these with the group. For example, Tana wants to \textit{learn} rune magic, Woody \textit{mistrusts it}, Basler \textit{is curious about it}, and Sir Emory \textit{is ignorant of it}.

Finally, look at your positions as a group, and ask yourselves, "how will my character interact with these other characters, given these positions?"

If the group talks about an idea they want to add to the world, but can't make it work as a pole star, that's okay. Just add it to the fictional world anyway.

\vfill\null % if the text isn't compressed enough, use this to "fill" the rest of the column

\columnbreak

% now we put in some vertical space, which depends on your font and title, to account for the overlapping minipage. Note the dot.
{\color{white}.}
\vspace{19mm}

\section{What is a good Pole Star?}

\textit{Pole stars should be accessible to the characters.} If there's dragons, sooner or later the characters should probably meet dragons. Preferably sooner.

\textit{Pole star positions should be personal.} Is there something in your character's history that makes them feel this way? Did an inciting incident give them strong feelings about it? Did a dragon burn their farmlands, save their sister, or impress them with its grandeur?

\textit{Not everyone should have the same position.} When everyone is opposed to "the Northern invasion" because invasions are bad, it shouldn't be a pole star.

\textit{Starting positions shouldn't immediately create strife.} If one character wants to save dragons and another wants to kill them, either rethink the positions or rethink dragons. If this kind of conflict happens later in the game and is dramatically interesting, though, then it's okay.

\section{Changing Pole Stars}

Pole stars that don't do their job - drive drama and create interactions - should be replaced with ones that do. Many people don't realize what their characters or storylines should really be about until they inhabit them for a little while. A pole star is a tool, not an unbreakable commitment.

If an idea for a pole star doesn't work out, look for an adjacent issue to replace it. For example, maybe there's a neutral kingdom in the path of an invasion, and some characters want to leave it alone while others want to convince it to join the fight. That kingdom then becomes a pole star instead.

Pole stars and positions can and should change in play. The players will discover new things they care about. Characters might take new positions on existing issues. For example, if the Northern invasion is thwarted in-game, players might revise their positions based on what happened, or the group can choose a new pole star.

\vfill\null % if the text isn't compressed enough, use this to "fill" the rest of the column

\columnbreak

\section{Example Pole Stars and Positions}

I want to (protect, manipulate, befriend, get help from) \textbf{the young prince}.

I want to (join, resist, investigate, expose) a \textbf{new religious order}.

I want to (research, exploit, explore, warn the world about) the \textbf{elemental imbalance}.

I want to (obtain, study, use, safeguard) the \textbf{magical artifact}.

I am (suspicious, excited, curious, angry) about the \textbf{new technological discovery}.

I (am skeptical of, want to believe in, follow) the \textbf{strange prophet}.

I want to (win a game, upstage a rival, impress someone, keep people safe, denounce antiquated traditions) during \textbf{a competitive event}.

\section{Random Pole Stars}

Your pole star is... (roll a d6)

\begin{enumerate}
\item [1:] a person or creature (e.g. a princess)
\item [2:] a group or organization (e.g. a fraternity)
\item [3:] an item or artifact (e.g. a ring of power)
\item [4:] a place (e.g. a lost island)
\item [5:] an event (e.g. an invasion)
\item [6:] a phenomenon (e.g. weather)
\end{enumerate}

And it's... (roll a d6)

\begin{enumerate}
\item [1:] tied to magic or religion
\item [2:] physically or socially powerful
\item [3:] recently arrived or departed the area
\item [4:] in or causing trouble for someone
\item [5:] risky or uncertain
\item [6:] strange or mysterious
\end{enumerate}

\section{Credits}

Thanks to Drew, Jay, Mike, and Deanna for playtesting help and suggestions.
Pole stars were inspired by the plot of "The Dragon Prince" on Netflix.

% this ends our multicols
\end{multicols}

% this ends our document
\end{document}
